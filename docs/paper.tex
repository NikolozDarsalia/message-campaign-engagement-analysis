\documentclass[12pt]{article}

\usepackage{amsmath, amssymb, amsthm}
\usepackage{setspace}
\usepackage{graphicx}
\usepackage{booktabs}
\usepackage{hyperref}
\usepackage{enumitem}

\setlength{\parskip}{0.9em}
\setlength{\parindent}{0pt}
\onehalfspacing

\title{Information Overload, Cognitive Capacity, and Customer Engagement:\\
Evidence from Retail Messaging Data}
\author{}
\date{}

\begin{document}

\maketitle

\section{Introduction and Major Research Question}

Digital retail platforms rely heavily on repeated messaging---emails, push notifications, and in-app messages---to stimulate customer engagement. Yet these messages arrive in an environment where customers face limited attention and must make repeated decisions under uncertainty: whether opening the next message will be worth the cognitive cost.

This paper studies how different sources of information pressure---including a firm’s own messaging frequency, competitor-side market pressure, contextual timing (weekday, hour, holidays), and major news events---affect customers’ cognitive capacity, fatigue, and ultimately engagement decisions.

\textbf{Central Research Question:}

\begin{quote}
How do multiple information-environment factors shape customer cognitive capacity and thereby influence message-opening behavior in retail settings?
\end{quote}

A motivating idea is that attention is scarce. When customers experience high information pressure---whether from the firm, from competitors, or from the external environment---their ability to process additional commercial messages declines. This results in message fatigue, lower engagement, and potentially market failure where firms collectively oversupply commercial information.

We also ask a second, practical question:

\begin{quote}
How do these effects differ across customer segments (loyal, occasional, dormant), and how should firms adjust messaging strategies for each group?
\end{quote}

These questions are directly relevant for retail campaign planning. Ignoring information pressure risks inefficient campaigns, reduced ROI, and long-run habituation effects.

\section{Conceptual Framework: Decision Under Uncertainty}

We model customer message engagement as a decision under uncertainty.

Customer $i$ at time $t$ chooses action $a_{it} \in \{0,1\}$ where $1=$ open, $0=$ ignore. Utility from opening:

\[
U_{it}(1) = E_{it}[V_{it}] - C_{it},
\]
\[
U_{it}(0) = 0.
\]

Decision rule:

\[
a_{it} = 1 \iff E_{it}[V_{it}] - C_{it} \ge 0.
\]

\subsection*{Expected Value}

\[
E_{it}[V_{it}] = \theta_i \cdot q_{it}
\]

\begin{itemize}[leftmargin=*]
\item $\theta_i$: latent preference / inherent propensity to value offers.
\item $q_{it}$: message-specific relevance signal.
\end{itemize}

\subsection*{Cognitive Cost and Fatigue}

\[
C_{it} = \alpha_i + \beta F_{it}
\]

\begin{itemize}[leftmargin=*]
\item $\alpha_i$: baseline cognitive cost.
\item $F_{it}$: fatigue state.
\item $\beta$: conversion factor from fatigue to cost.
\end{itemize}

Fatigue dynamics:

\[
F_{i,t} = \rho F_{i,t-1} + \gamma \cdot \text{msgs}_{i,t-1} 
        + \delta M_t + \varepsilon_{it}.
\]

\begin{itemize}[leftmargin=*]
\item $\text{msgs}_{i,t-1}$: recent firm messages.
\item $M_t$: market-wide or external information pressure.
\item $\rho \in [0,1]$: persistence.
\end{itemize}

Customers with high cognitive capacity behave almost risk-neutral; fatigued customers avoid messages even if potentially valuable.

This framework predicts heterogeneous responses across customer segments.

\section{Testable Predictions}

We derive three main predictions:

\begin{enumerate}
\item \textbf{Information pressure reduces engagement.}  
Higher messaging frequency (own or market-wide) lowers open probability by increasing cognitive cost.

\item \textbf{Heterogeneity across segments.}
    \begin{itemize}
        \item Loyal customers exhibit slower fatigue accumulation.
        \item Dormant customers show strong negative responses even at moderate frequency.
    \end{itemize}

\item \textbf{Context matters.}  
Timing (weekday, hour, holiday periods) shifts cognitive cost and open rates.
\end{enumerate}

\section{Data Plan}

We use fully engineered internal datasets.

\subsection*{Outcome Variable}
\begin{itemize}[leftmargin=*]
\item Message opened (binary)
\end{itemize}

\subsection*{Key Explanatory Variables}

\textbf{Own-firm information pressure}
\begin{itemize}[leftmargin=*]
\item Messages in last 1/7/30 days
\item Recency since last message
\item Frequency volatility
\item Fatigue ratios (short-term vs long-term)
\end{itemize}

\textbf{Market-wide information pressure}
\begin{itemize}[leftmargin=*]
\item Competitor or simulated market messaging averages
\item External news cycles (if available)
\end{itemize}

\textbf{Contextual cognitive-load shifters}
\begin{itemize}[leftmargin=*]
\item Day of week, time of day
\item Holiday indicator
\item Special events
\end{itemize}

\textbf{Expected-value proxies}
\begin{itemize}[leftmargin=*]
\item Customer historical open rates (1-week, 1-month, lifetime)
\item Subject line features, personalization, discount mentions
\item Campaign historical open/click rates
\item Customer value (visits, purchases)
\end{itemize}

\textbf{Customer segmentation (three groups)}
\begin{enumerate}
\item Loyal / high-engagement
\item Occasional
\item Dormant / low-engagement
\end{enumerate}

\section{Empirical Strategy}

We implement three components:

\subsection{1. Descriptive Exploration}
\begin{itemize}[leftmargin=*]
\item Open rate vs. messaging frequency
\item Fatigue curves across segments
\item Contextual timing patterns
\end{itemize}

\subsection{2. Reduced-Form Econometric Model}

We estimate:

\[
\Pr(\text{open}_{it}=1)
= 
\Lambda\big(
\beta_1 \text{Fatigue}_{it}
+ \beta_2 \text{ExpectedValue}_{it}
+ \beta_3(\text{Fatigue}\times\text{Segment})
+ X_{it}\gamma
\big),
\]

where $X_{it}$ includes timing controls and campaign fixed effects.

\subsection{3. Heterogeneous Effects \& Policy Simulations}
\begin{itemize}[leftmargin=*]
\item Estimate models separately by segment.
\item Simulate open probabilities under alternative messaging strategies.
\item Provide segment-specific recommendations.
\end{itemize}

\section{Expected Outputs}

\subsection*{Figures}
\begin{itemize}[leftmargin=*]
\item Fatigue curves for each segment
\item Heatmap: predicted open probability by (fatigue $\times$ expected value)
\item Timing-effect plots
\end{itemize}

\subsection*{Tables}
\begin{itemize}[leftmargin=*]
\item Model coefficients and marginal effects
\item Heterogeneous effects by segment
\item Simulation results for optimal messaging frequencies
\end{itemize}

\subsection*{Contribution}
\begin{itemize}[leftmargin=*]
\item Demonstrate how multiple information pressures jointly determine cognitive capacity.
\item Highlight strong heterogeneity across customer segments.
\item Provide actionable guidance for messaging under limited attention.
\end{itemize}

\section{Feasibility}

We already have:
\begin{itemize}[leftmargin=*]
\item Fatigue features, expected-value proxies, and market pressure features
\item Segmentation options
\item Clean panel data structure
\item Experience with logit/probit modeling and interactions
\end{itemize}

The analysis is fully feasible within the available timeline.

\section*{Conclusion}

This project combines behavioral economics, decision-making under uncertainty, and high-dimensional retail data to understand how information environments shape customer engagement---and how firms can design more effective, less intrusive communication strategies.

\end{document}
