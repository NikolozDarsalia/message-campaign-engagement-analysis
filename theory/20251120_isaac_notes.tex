Suggestion to write a timeline
open or not is based on some expected value(clarifying q from Isaac)
even before 
-we know we observe signal which is noisy from the truth
-you expect theta given signal
-care more about the variance
idea about Gaussian/signal components
-you get n free signals
-and then f firm signals 
-then the sum is the total that you have from fatigue -> unlike normal, there is a "constraint" on number of signals before fatigue
-you get a precision for all of the signals/cost of processing the signals
Referenced lecture slides in ch3 and ch4,
Ch4
-firm choses n projects (or messages) for time t, t+1
-the future rate is discounted at beta<1
-Using Nt projects -> cost rNt^2 (quadratic)
the output at t: A - (at-ajt*)^2
Aggregrate signal:
st = a* + (1/nt)*sigma(ejt)

TLDR: Use the static Gaussian-signal model from Chapters 3–4 of the slides, but add a constraint on the number of signals because sending too many emails creates fatigue. Mabye we don’t need the dynamic model.


Additional GPT notes:



